\documentclass{article}

\usepackage{arxiv}

\usepackage[utf8]{inputenc} % allow utf-8 input
\usepackage[T1]{fontenc}    % use 8-bit T1 fonts
\usepackage{hyperref}       % hyperlinks
\usepackage{url}            % simple URL typesetting
\usepackage{booktabs}       % professional-quality tables
\usepackage{amsfonts}       % blackboard math symbols
\usepackage{nicefrac}       % compact symbols for 1/2, etc.
\usepackage{microtype}      % microtypography
\usepackage{lipsum}

\title{A template for the \emph{arxiv} style}


\author{
  Jiaming Lai\\
  % \thanks{Use footnote for providing further
  %   information about author (webpage, alternative
  %   address)---\emph{not} for acknowledging funding agencies.} \\
  Department of Electrical and Computer Engineering\\
  University of California San Diego\\
  La Jolla, CA 92037 \\
  \texttt{jil136@ucsd.edu} \\
  %% examples of more authors
  %% \AND
  %% Coauthor \\
  %% Affiliation \\
  %% Address \\
  %% \texttt{email} \\
  %% \And
  %% Coauthor \\
  %% Affiliation \\
  %% Address \\
  %% \texttt{email} \\
  %% \And
  %% Coauthor \\
  %% Affiliation \\
  %% Address \\
  %% \texttt{email} \\
}

\begin{document}
\maketitle

% \begin{abstract}
%   We use 
% \end{abstract}
% % keywords can be removed
% \keywords{Eigenface \and Reconstruction}

\section{Introduction}
Following the "Eigenface" technique proposed by Matthew, et al.\ (1995)
\cite{eigenface1}\cite{eigenface2}, we derive eigenfaces (obtained via PCA) from face image
dataset. We use these eigenfaces (also known as principal components) to reconstruct different
kinds of images and evaluate the result via computing MSE. In section 2, we will cover a simplified
description of how the PCs are calculated and reconstruction is done. Combining with the singular 
values plot, we will give our justification of those PCs. Section 3-5 are about reconstruction result
using images from 190 individuals’ neutral expression image set, from 190 individuals’ smiling expression
image set and from the other 10 individuals’ neutral expression image set. In section 6, we will reconstruct
non-human images using all the PCs and evaluate the result. Section 7 is to reconstruct rotated images
and our comment of the result.

\section{Singular values plot and justification of PCs}
% \label{sec:headings}
% \lipsum[4] See Section \ref{sec:headings}.
\subsection{Simplified description of PC computation and reconstruction}

\subsection{Singular values plot}
The following figure is the singular values plot from data matrix.

\subsection{Justification and explaination}

\section{Examples of citations, figures, tables, references}
\label{sec:others}
\lipsum[8] \cite{kour2014real,kour2014fast} and see \cite{hadash2018estimate}.

The documentation for \verb+natbib+ may be found at
\begin{center}
  \url{http://mirrors.ctan.org/macros/latex/contrib/natbib/natnotes.pdf}
\end{center}
Of note is the command \verb+\citet+, which produces citations
appropriate for use in inline text.  For example,
\begin{verbatim}
   \citet{hasselmo} investigated\dots
\end{verbatim}
produces
\begin{quote}
  Hasselmo, et al.\ (1995) investigated\dots
\end{quote}

\begin{center}
  \url{https://www.ctan.org/pkg/booktabs}
\end{center}


\subsection{Figures}
\lipsum[10] 
See Figure \ref{fig:fig1}. Here is how you add footnotes. \footnote{Sample of the first footnote.}
\lipsum[11] 

\begin{figure}[h]
  \centering
  \fbox{\rule[-.5cm]{4cm}{4cm} \rule[-.5cm]{4cm}{0cm}}
  \caption{Sample figure caption.}
  \label{fig:fig1}
\end{figure}

\subsection{Tables}
\lipsum[12]
See awesome Table~\ref{tab:table}.

\begin{table}
 \caption{Sample table title}
  \centering
  \begin{tabular}{lll}
    \toprule
    \multicolumn{2}{c}{Part}                   \\
    \cmidrule(r){1-2}
    Name     & Description     & Size ($\mu$m) \\
    \midrule
    Dendrite & Input terminal  & $\sim$100     \\
    Axon     & Output terminal & $\sim$10      \\
    Soma     & Cell body       & up to $10^6$  \\
    \bottomrule
  \end{tabular}
  \label{tab:table}
\end{table}

\subsection{Lists}
\begin{itemize}
\item Lorem ipsum dolor sit amet
\item consectetur adipiscing elit. 
\item Aliquam dignissim blandit est, in dictum tortor gravida eget. In ac rutrum magna.
\end{itemize}


\bibliographystyle{unsrt}  
%\bibliography{references}  %%% Remove comment to use the external .bib file (using bibtex).
%%% and comment out the ``thebibliography'' section.
\bibliography{references}

\end{document}
